\documentclass[12pt,letterpaper,reqno]{amsart}
\setlength{\oddsidemargin}{.0in}
\setlength{\evensidemargin}{.0in}
\setlength{\textwidth}{6.5in}
\setlength{\topmargin}{-.3in}
\setlength{\headsep}{.20in}
\setlength{\textheight}{9.in}
\usepackage{amsmath}
\usepackage{amsfonts}
\usepackage{amssymb}
\usepackage{amsthm}
\usepackage[all]{xy}
\usepackage{graphicx}

\usepackage{setspace}
\doublespace


%Here are some user-defined notations
\newcommand{\E}{\mathbb E}
\newcommand{\R}{\mathbb R}

%improving spacing in tables (space above and below characters in a row)
\newcommand{\tfix}{\rule{0pt}{2.6ex}}
\newcommand{\bfix}{\rule[-1.2ex]{0pt}{0pt}}


%Here are commands with variable inputs 
\newcommand{\intf}[1]{\int_a^b{#1}\,dx}
\newcommand{\intfb}[3]{\int_{#1}^{#2}{#3}\,dx}
\newcommand{\pln}[1]{$\sm${\tt #1}}
\newcommand{\bgn}[1]{$\tt {\sm}begin\{#1\}$}
\newcommand{\nd}[1]{$\tt {\sm}end\{#1\}$}
\newcommand{\marginalfootnote}[1]{%
        \footnote{#1}
        \marginpar[\hfill{\sf\thefootnote}]{{\sf\thefootnote}}}
\newcommand{\edit}[1]{\marginalfootnote{#1}}


\newtheorem*{wellorder*}{Well Ordering Principle}
%These commands deal with theorem-like environments (i.e., italic)
\theoremstyle{plain}
\newtheorem{theorem}{Theorem}[section]
\newtheorem{corollary}[theorem]{Corollary}
\newtheorem{lemma}[theorem]{Lemma}
\newtheorem{conjecture}[theorem]{Conjecture}
%\newtheorem{wellorder}[theorem]{Well Ordering Principle}
%\newtheorem{equation}{Theorem}[section]
%These deal with definition-like environments (i.e., non-italic)
\theoremstyle{definition}
\newtheorem{definition}[theorem]{Definition}
\newtheorem{example}[theorem]{Example}
\newtheorem{remark}[theorem]{Remark}

\theoremstyle{definition}
\newtheorem{Exercise}{Exercise 2.}
\newtheorem{exercise}{exercise 2.}

%This numbers equations by section
\numberwithin{equation}{section}
%\numberwithin{equation}{Theorem}[section]


%This is for hypertext references
\usepackage{color}
\usepackage{hyperref}




\begin{document}
\author{Zhizhong Pu}
\title{Solutions - Chapter 2}
\date{Feb 2023}
\maketitle

\thispagestyle{empty}

%% 2.1
\begin{Exercise} Find $\E[\E[\E[Y|X_1,X_2,X_3]|X_1,X_2]|X_1]$.

    Since $\E[\E[Y|X_1,X_2,X_3]|X_1,X_2]=\E[Y|X_1,X_2]$ by LIE, then we have:
    \[
    \E[\E[\E[Y|X_1,X_2,X_3]|X_1,X_2]|X_1] = \E[\E[Y|X_1,X_2]|X_1] = \E[Y|X_1]
    \]
    
\end{Exercise}

%% 2.2
\begin{Exercise} If $\E[Y|X]=a+bX$, find $\E[YX]$ as a function of moments of $X$.
    Since $\E[Y|X]=a+bX$, we follow chapter 2.8 to define CEF error:
    \[
    e=Y-\E[Y|X]=Y-a+bX\Rightarrow Y=a+bX+e
    \]
    Then substitute $Y$ into $\E[YX]$:
    \[
    \E[YX]=\E[(a+bX+e)X]=a\E[X]+b\E[X^2]+\E[eX]
    \]
    Note that the last term is $0$ by Theorem 2.4.4, then
    \[
     \E[YX]=\E[(a+bX+e)X]=b\E[X^2]+a\E[X]
    \]

\end{Exercise}

%% 2.3 
\begin{Exercise}
Prove Theorem 2.4.4 using the law of iterated expectations: for any function $h(x)$ such that $\E|h(X)e|<\infty$ then $\E[h(X)e]=0$
    \begin{proof} Wooldridge (2010) CE.6 Proof

    Given any function $h()$, we can write
    \[
    \E[h(X)e]=\E[\E(h(X)e|X)]=\E[h(X)\E(e|X)]=\E[h(X)0]=0
    \]
    using LIE, take-out-what's-known property, and $\E(e|X)=0$ from Theorem 2.4.1, respectively.
    \end{proof}
\end{Exercise}

%% 2.4
\begin{Exercise} See book

By the definition of conditional expectation:
\[
    \E[Y|X=0]=\sum_{y\in{0,1}}yP(Y=y|X=0)=1P(Y=1|X=0)=0.8 
\]
\[
    \E[Y|X=1]=\sum_{y\in{0,1}}yP(Y=y|X=1)=1P(Y=1|X=1)=0.6
\]
(Note that LIE can be verified: $\E[\E[Y|X]]=0.7=\E[Y]$)

By LOTUS:
\[
    \E[Y^2|X=0]=\sum_{y\in{0,1}}y^2P(Y=y|X=0)=1P(Y=1|X=0)=0.8 \\
\]
\[
    \E[Y^2|X=1]=\sum_{y\in{0,1}}y^2P(Y=y|X=1)=1P(Y=1|X=1)=0.6
\]

By definition of conditional variance:
\[
    Var(Y|X=0)=\E[Y^2|X=0]-(\E[Y|X=0])^2=0.8-(0.8)^2=0.16
\]
\[
    Var(Y|X=1)=\E[Y^2|X=1]-(\E[Y|X=1])^2=0.6-(0.6)^2=0.24
\]
\end{Exercise}


%% 2.5
\begin{Exercise} Show that $\sigma^2(X)$ is the best predictor of $e^2$ given $X$

\end{Exercise} 

%% 2.6

\begin{Exercise} Use $Y=m(X)+e$ to show that $Var(Y)=Var[(m(X)]+\sigma^2$. 

*Note on notation: $\sigma^2=\E[e^2]$. 

Consider the LHS. By definition of variance:
\[
    Var(Y)=Var[m(X)+e]=\E[(m(X)+e)^2]-(\E[m(X)+e])^2
\]
The first term is (using $\E[m(X)e]=0$ from Theorem 2.4.4.):
\[
    \E[(m(X)+e)^2]=\E[m^2(X)]+2\E[m(X)e]+\E[e^2]=\E[m^2(X)]+\E[e^2]
\]
The second term is (using 0 unconditional mean of error term):
\[
    -(\E[m(X)+e])^2=-(\E[m(X)]+\E[e])^2=-(\E[m(X)]+0)^2=-\E[m(X)]^2
\]
Sum up the two terms:
\[
    Var(Y)=\E[(m(X)+e)^2]-(\E[m(X)+e])^2=\E[m^2(X)]+\E[e^2]-\E[m(X)]^2=Var[(m(X)]+\sigma^2
\]


\end{Exercise} 


%% 2.7

\begin{Exercise} Show that the conditional variance can be written as $\sigma^2(X)=\E[Y^2|X]-\E[Y|X]^2$

By definition 2.1, we can write
\[\begin{split}
    \sigma^2(X) & \equiv Var(Y|X) = \E[(Y-\E(Y|X))^2|X] \\
    & = \E[Y^2-2Y\E(Y|X)+\E(Y|X)^2|X] \\
    & = \E[Y^2|X] - 2\E[Y\E(Y|X)|X] + \E[\E(Y|X)^2|X]
\end{split}\]

Note that in the last two terms, $\E(Y|X)$ and $\E(Y|X)^2$ are both functions of X, so we can apply take-out-what's-known property:
\[
    = \E[Y^2|X] - 2\E(Y|X)\E(Y|X) + \E(Y|X)^2\E[1|X] = \E[Y^2|X] - \E(Y|X)^2
\]
    
\end{Exercise}

% 2.8
\begin{Exercise}
\end{Exercise}

% 2.9
\begin{Exercise}
\end{Exercise}

% 2.10
\begin{Exercise} True or False. If $Y = X\beta+e$, $X \in \R$, and $\E[e|X] = 0$, then $\E[X^2e]=0$.
    True. 
    \[
        \E[X^2e]=\E[\E[X^2e|X]]=\E[X^2\E[e|X]]=\E[0X^2]=0
    \]
\end{Exercise}

% 2.11
\begin{Exercise}
\end{Exercise}

% 2.12
\begin{Exercise}
\end{Exercise}

% 2.13
\begin{Exercise}
\end{Exercise}

% 2.14
\begin{Exercise}
\end{Exercise}

% 2.15

\begin{Exercise} Consider the intercept-only model $Y = a + e$ with $a$ the best linear predictor. Show that
$a = \E[Y]$. 

We can write MSE as:
\[\begin{split}
    \E[e^2] =  \E[(Y-a)^2] = \E[Y^2]-2a\E[Y]+a^2
\end{split}\]
Apply the first order condition:
\[
    \frac{\partial}{\partial a} \E[e^2] = -2\E[Y]+2a \equiv 0
\]
This shows $a = \E[Y]$. 
\end{Exercise}

% 2.16
\begin{Exercise}

I do not recommend this problem for calculations are complicated. 


\[
    \beta=E(XX')^{-1}(XY)=\frac{\E(XY)}{\E(X^2)}=\frac{7}{16} \cdot \frac{15}{7} = \frac{15}{16} 
\]

\[
    \E[Y|X]= \int_{0}^{1} y \cdot f_{Y|X}(y|x) \,dy = \int_{0}^{1} y \frac{x^2+y^2}{x^2+\frac{1}{3}} \,dy = \frac{6x^2+3}{12x^2+4}
\]

\end{Exercise}

\end{document}	
